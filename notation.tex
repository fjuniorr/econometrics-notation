\documentclass[11pt,dvips,a4paper]{article}
\usepackage{amsfonts}
\usepackage{amssymb}
\usepackage{graphics}
\usepackage{epsfig}
\usepackage{verbatim}
\usepackage{bm}
\usepackage{latexsym}
\usepackage{amsmath}
%\usepackage{endnotes}
%\usepackage{makeidx}
%\usepackage[dvips]{epsfig}
\setlength{\textwidth}{32pc}
\setlength{\textheight}{197mm}
%%%
\ProvidesPackage{ee}

\RequirePackage{amsfonts}
\RequirePackage{amssymb}
\RequirePackage{graphics}
\RequirePackage{epsfig}
\RequirePackage{verbatim}
\RequirePackage{bm}
\RequirePackage{latexsym}
\RequirePackage{amsmath}

%%This is style file ee.sty.
%%version 30/01/2002
%
\newcommand{\newoperator}[3]{\newcommand*{#1}{\mathop{#2}#3}}
\newcommand{\renewoperator}[3]{\renewcommand*{#1}{\mathop{#2}#3}}
%
% symbols C,N,Q,R,Z for sets
\newcommand{\SC}{\mathbb{C}}
\newcommand{\SN}{\mathbb{N}}
\newcommand{\SQ}{\mathbb{Q}}
\newcommand{\SR}{\mathbb{R}}
\newcommand{\SZ}{\mathbb{Z}}
%
% calligraphic capital letters
\newcommand{\calA}{\mathcal{A}}
\newcommand{\calB}{\mathcal{B}}
\newcommand{\calC}{\mathcal{C}}
\newcommand{\calD}{\mathcal{D}}
\newcommand{\calE}{\mathcal{E}}
\newcommand{\calF}{\mathcal{F}}
\newcommand{\calG}{\mathcal{G}}
\newcommand{\calH}{\mathcal{H}}
\newcommand{\calI}{\mathcal{I}}
\newcommand{\calJ}{\mathcal{J}}
\newcommand{\calK}{\mathcal{K}}
\newcommand{\calL}{\mathcal{L}}
\newcommand{\calM}{\mathcal{M}}
\newcommand{\calN}{\mathcal{N}}
\newcommand{\calO}{\mathcal{O}}
\newcommand{\calP}{\mathcal{P}}
\newcommand{\calQ}{\mathcal{Q}}
\newcommand{\calR}{\mathcal{R}}
\newcommand{\calS}{\mathcal{S}}
\newcommand{\calT}{\mathcal{T}}
\newcommand{\calU}{\mathcal{U}}
\newcommand{\calV}{\mathcal{V}}
\newcommand{\calW}{\mathcal{W}}
\newcommand{\calX}{\mathcal{X}}
\newcommand{\calY}{\mathcal{Y}}
\newcommand{\calZ}{\mathcal{Z}}
%
% bold lowercase and capital letters for vectors (v) and matrices (m)
\newcommand{\mA}{\bm A}
\newcommand{\va}{\bm a}
\newcommand{\mB}{\bm B}
\newcommand{\vb}{\bm b}
\newcommand{\mC}{\bm C}
\newcommand{\vc}{\bm c}
\newcommand{\mD}{\bm D}
\newcommand{\vd}{\bm d}
\newcommand{\mE}{\bm E}
\newcommand{\ve}{\bm e}
\newcommand{\mF}{\bm F}
\newcommand{\vf}{\bm f}
\newcommand{\mG}{\bm G}
\newcommand{\vg}{\bm g}
\newcommand{\mH}{\bm H}
\newcommand{\vh}{\bm h}
\newcommand{\mI}{\bm I}
\newcommand{\vi}{\bm i}
\newcommand{\mJ}{\bm J}
\newcommand{\vj}{\bm j}
\newcommand{\mK}{\bm K}
\newcommand{\vk}{\bm k}
\newcommand{\mL}{\bm L}
\newcommand{\vl}{\bm l}
\newcommand{\mM}{\bm M}
\newcommand{\vm}{\bm m}
\newcommand{\mN}{\bm N}
\newcommand{\vn}{\bm n}
\newcommand{\mO}{\bm O}
\newcommand{\vo}{\bm o}
\newcommand{\mP}{\bm P}
\newcommand{\vp}{\bm p}
\newcommand{\mQ}{\bm Q}
\newcommand{\vq}{\bm q}
\newcommand{\mR}{\bm R}
\newcommand{\vr}{\bm r}
\newcommand{\mS}{\bm S}
\newcommand{\vs}{\bm s}
\newcommand{\mT}{\bm T}
\newcommand{\vt}{\bm t}
\newcommand{\mU}{\bm U}
\newcommand{\vu}{\bm u}
\newcommand{\mV}{\bm V}
\newcommand{\vv}{\bm v}
\newcommand{\mW}{\bm W}
\newcommand{\vw}{\bm w}
\newcommand{\mX}{\bm X}
\newcommand{\vx}{\bm x}
\newcommand{\mY}{\bm Y}
\newcommand{\vy}{\bm y}
\newcommand{\mZ}{\bm Z}
\newcommand{\vz}{\bm z}
%
% bold Greek lowercase letters for vectors (v)
\newcommand{\valpha}{\bm \alpha}
\newcommand{\vbeta}{\bm \beta}
\newcommand{\vgamma}{\bm \gamma}
\newcommand{\vdelta}{\bm \delta}
\newcommand{\vepsi}{\bm \epsi}
\newcommand{\vvarepsilon}{\bm \varepsilon}
\newcommand{\vzeta}{\bm \zeta}
\newcommand{\veta}{\bm \eta}
\newcommand{\vtheta}{\bm \theta}
\newcommand{\viota}{\bm \iota}
\newcommand{\vkappa}{\bm \kappa}
\newcommand{\vlambda}{\bm \lambda}
\newcommand{\vmu}{\bm \mu}
\newcommand{\vnu}{\bm \nu}
\newcommand{\vxi}{\bm \xi}
\newcommand{\vpi}{\bm \pi}
\newcommand{\vrho}{\bm \rho}
\newcommand{\vsigma}{\bm \sigma}
\newcommand{\vtau}{\bm \tau}
\newcommand{\vupsilon}{\bm \upsilon}
\newcommand{\vphi}{\bm \phi}
\newcommand{\vchi}{\bm \chi}
\newcommand{\vpsi}{\bm \psi}
\newcommand{\vomega}{\bm \omega}
%
% bold Greek capital letters for matrices (m)
\newcommand{\mGamma}{\bm \varGamma}
\newcommand{\mDelta}{\bm \varDelta}
\newcommand{\mTheta}{\bm \varTheta}
\newcommand{\mLambda}{\bm \varLambda}
\newcommand{\mXi}{\bm \varXi}
\newcommand{\mPi}{\bm \varPi}
\newcommand{\mSigma}{\bm \varSigma}
\newcommand{\mUpsilon}{\bm \varUpsilon}
\newcommand{\mPhi}{\bm \varPhi}
\newcommand{\mPsi}{\bm \varPsi}
\newcommand{\mOmega}{\bm \varOmega}
%
% roman letters in mathematics
\newcommand{\rb}{\ensuremath{\mathrm{b}}}
\newcommand{\rB}{\ensuremath{\mathrm{B}}}
\newcommand{\rC}{\ensuremath{\mathrm{C}}}
\newcommand{\rD}{\ensuremath{\mathrm{D}}}
\newcommand{\rf}{\ensuremath{\mathrm{f}}}
\newcommand{\rF}{\ensuremath{\mathrm{F}}}
\newcommand{\rH}{\ensuremath{\mathrm{H}}}
\newcommand{\rL}{\ensuremath{\mathrm{L}}}
\newcommand{\rN}{\ensuremath{\mathrm{N}}}
\newcommand{\rt}{\ensuremath{\mathrm{t}}}
\newcommand{\rU}{\ensuremath{\mathrm{U}}}
\newcommand{\rGam}{\ensuremath{\mathrm{Gam}}}
\newcommand{\rBeta}{\ensuremath{\mathrm{Beta}}}
%
\newcommand{\Bin}{\ensuremath{\mathrm{Bin}}}
\newcommand{\eu}{\ensuremath{\mathrm{e}}}
\newcommand{\iu}{\ensuremath{\mathrm{i}}}
\newcommand{\LN}{\ensuremath{\mathrm{LN}}}
\newcommand{\IN}{\ensuremath{\mathrm{IN}}}

\newcommand{\Poi}{\ensuremath{\mathrm{Poi}}}
%
\newcommand{\ped}[1]{\ensuremath{_\mathrm{#1}}} %pedex
\newcommand{\ap}[1]{\ensuremath{^\mathrm{#1}}} %apex
\renewoperator{\Re}{\mathrm{Re}}{\nolimits}
\renewoperator{\Im}{\mathrm{Im}}{\nolimits}
%
% letters for (partial) differentiation
%\newcommand{\rd}{\ensuremath{\mathrm{d}}}
\makeatletter
\newcommand{\rd}{\@ifnextchar^{\DIfF}{\DIfF^{}}}
\def\DIfF^#1{%
   \mathop{\mathrm{\mathstrut d}}%
   \nolimits^{#1}\gobblespace}
\def\gobblespace{\futurelet\diffarg\opspace}
\def\opspace{%
   \let\DiffSpace\!%
   \ifx\diffarg(%
   \let\DiffSpace\relax
   \else
   \ifx\diffarg[%
   \let\DiffSpace\relax
   \else
   \ifx\diffarg\{%
   \let\DiffSpace\relax
   \fi\fi\fi\DiffSpace}
\newcommand{\deriv}[3][]{\frac{\rd^{#1}#2}{\rd #3^{#1}}}
\newcommand{\pderiv}[3][]{\frac{\partial^{#1}#2}{\partial #3^{#1}}}
%
% operatornames
\newcommand{\bias}{\operatorname{bias}}
\newcommand{\col}{\operatorname{col}}
\newcommand{\corr}{\operatorname{corr}}
\newcommand{\cov}{\operatorname{cov}}
\newcommand{\dg}{\operatorname{dg}}
\newcommand{\diag}{\operatorname{diag}}
\newcommand{\E}{\operatorname{E}}
\newcommand{\etr}{\operatorname{etr}}
\newoperator{\ip}{\mathrm{int}}{\nolimits}
\newcommand{\kur}{\operatorname{kur}}
%\newcommand{\median}{\operatorname{med}}
\newcommand{\MSE}{\operatorname{MSE}}
\newcommand{\plim}{\operatorname{plim}}
\newcommand{\rk}{\operatorname{rk}}
\newcommand{\sgn}{\operatorname{sgn}}
\newcommand{\tr}{\operatorname{tr}}
\newcommand{\var}{\operatorname{var}}
\renewcommand{\vec}{\operatorname{vec}}
\newcommand{\vech}{\operatorname{vech}}
%
% other definitions
\newcommand{\distr}{\sim}
\newcommand{\adistr}{\stackrel{a}{\distr}}
\newcommand{\diff}{\Delta}
\newcommand{\fordiff}{\bigtriangleup}
\newcommand{\fdiff}{\diff_{\rf}}
\newcommand{\bdiff}{\diff_{\rb}}
%
%\mathchardef\varepsilon="010F
%\mathchardef\epsilon="0122
%\mathchardef\eps="010F
\newcommand{\eps}{\epsilon}
\newcommand{\epsi}{\varepsilon}
%
\newcommand{\longto}{\longrightarrow}
\newcommand{\pto}{\stackrel{p}{\longrightarrow}}
\newcommand{\dto}{\stackrel{d}{\longrightarrow}}
\newcommand{\wto}{\stackrel{w}{\longrightarrow}}
%
\newcommand{\Infmat}{\bm\calI}
\newcommand{\Hesmat}{\bm\calH}
\newcommand{\bcdot}{\raisebox{1pt}{\textbf{\large .}}}
\newcommand{\interior}[1]{\overset{\circ}{#1}}
%
\newcommand{\vones}{\bm\imath}
\newcommand{\vzeros}{\boldsymbol{0}}
\newcommand{\mZeros}{\mathbf{O}}
%
% additional commands
\newcommand{\EE}[1]{\E\left(#1\right)}
\newcommand{\prob}[1]{\Pr\left(#1\right)}
%

\endinput
%%
\newcommand{\hspacesymbols}%
   {$\{x: x \in S, x$ satisfies $P\}\;\;$} %symbols index
\newcommand{\type}[1]{{\tt$\backslash$#1}}
%
\renewcommand{\thepage}{\hfill \arabic{page} \hfill}
\renewcommand{\thesection}{\arabic{section}}
\renewcommand{\theequation}{\arabic{equation}}
\pagestyle{plain}
\markright{}
\renewcommand{\baselinestretch}{1.20}
\pagenumbering{arabic}

\begin{document}
\begin{center}
\huge
{\bf Notation in Econometrics: \\A Proposal for a Standard}\\
\ \\
\Large
Karim M. Abadir   \\
{\it Department of Mathematics and Department of \\
Economics, University of York, York, UK} \\
E-mail: kma4@york.ac.uk \\
\ \\
Jan R. Magnus\\
{\it CentER, Tilburg University, Tilburg, The Netherlands} \\
E-mail: magnus@uvt.nl \\
\ \\
%\today
\emph{The Econometrics Journal}, 5 (2002), 76--90.\\
\end{center}
\ \\
\ \\
\normalsize
\noindent
{\it Summary:} This paper proposes a standard for notation in econometrics.
It presents a fully integrated and internally consistent
framework for notation and abbreviations, which is as close as
possible to existing common practice and also obeys ISO regulations.
The symbols used are instantly recognizable and interpretable,
thus minimizing ambiguity and enhancing reading efficiency.
The standard is designed in a flexible manner,
thus allowing for future extensions. \\
\ \\
{\it Key words:} Notation, Symbols, Econometrics, International
Organization for Standardization (ISO). \\
\ \\
{\it JEL Code:} C10
\setcounter{equation}{0}
\newpage
\section{Introduction}
Few things are as boring as questions of notation. Serious researchers
should do serious research and not waste their time thinking about notation.
The mathematician J.E. Littlewood said about Jordan that if he (Jordan) had
four things on the same footing (such as $a$, $b$, $c$, $d$) they would
appear as
$$
a,\quad
M_3',\quad
\epsi_2,\quad
\Pi''_{1,2};
$$
see Bollob\'{a}s (1986, p. 60).

On the other hand, many serious researchers {\it did\/} worry
about notation.
Jan Tinbergen propagated that
`when you have an index to a certain variable you should use
the capital letter as its upper limit.' For example,
$i=1,\dots,I$ and $j=1,\dots,J$, because this
`was just a little detail that could help you a lot
to see through things' (Magnus and Morgan, 1987, p. 127).

In physics, engineering, and chemistry a serious attempt
has been made to standardize symbols.
The International Organization for Standardization (ISO) has published
international regulations (ISO Standards Handbook, 1982)
and the International Union of Pure and Applied Physics (IPU)
has issued recommendations (CRC Handbook of Chemistry and Physics, 1988).
These regulations are generally followed by the profession, with one major
exception: the treatment of lowercase single-letter constants
(such as the base of natural logarithms $\eu$ and the imaginary unit
$\iu$\ --- very often written as $e$ and $i$, contrary to ISO regulations)
or operators (such as the derivative operator $\rd$ --- often
written as $d$).\footnote{See Beccari (1997) for further discussion
and some \LaTeX\ tricks for physicists and engineers.}
It appears that the profession finds that single-letter lowercase roman
mathematical symbols look odd. There are examples of this
phenomenon in econometrics too: one often sees
$\det(A)$ for determinant, $\E(x)$ for expectation,
but $r(A)$ for rank.

Notation matters. A good and consistent notation helps in the
understanding, communication and development of our profession.
In the Renaissance, mathematics was written in a verbal style with
p for plus, m for minus and R for square root. So, when Cardano
(1501--1576) writes
\begin{gather*}
5\text{p}:\text{Rm}:15\\
5\text{m}:\text{Rm}:15\\
25\text{m}:\text{m}:15\;\text{qd}\;\text{est}\;40,
\end{gather*}
he means
$(5+\sqrt{-15})(5-\sqrt{-15})=25-(-15)=40$,
see Kline (1972, p. 260).
There is no doubt that the development of good notation has been
of great importance in the history of mathematics.

In this paper we attempt to harmonize the various practices in econometrics
notation. It proposes a fully integrated and internally consistent
framework for notation and abbreviations, which is as close as
possible to existing common practice and also obeys ISO regulations.
The symbols used are instantly recognizable and interpretable,
thus minimizing ambiguity and enhancing reading efficiency.
Using a common notation will save authors the effort to define
their notation in every paper.
Only special notation needs to be defined.
We have tried to design our standard in a flexible manner,
allowing for future extensions in specialized fields.

There are many problems in designing a consistent notation.
Our hope is to provide a useful benchmark
and starting point for an evolving process.
The notation is \LaTeX\ oriented.
Many \LaTeX\ definitions are provided,
and the complete list of definitions can be downloaded from
{\it http:/\!/cwis.kub.nl/$\sim$few5/center/staff/magnus/notation.htp}.
%
\section{Vectors and matrices}
Vectors are lowercase and matrices are uppercase symbols.
Moreover, both vectors and matrices are written in bold-italic.
The vectors $\va$, $\vb$,\dots, $\vz$ are produced by
\type{va}, \type{vb},\dots, \type{vz},
and the matrices $\mA$, $\mB$,\dots, $\mZ$ by
\type{mA}, \type{mB},\dots, \type{mZ}.

Vectors can also be denoted by Greek lowercase letters:
$\valpha$,\dots, $\vomega$
(\type{valpha},\dots, \type{vomega}),
and matrices by Greek uppercase letters,
such as $\mGamma$ (\type{mGamma})
or $\mTheta$ (\type{mTheta}).

We write
$$
\va=
\begin{pmatrix}
a_1\\
a_2\\
\vdots\\
a_n
\end{pmatrix},
\qquad
\mA=
\begin{pmatrix}
a_{11} & a_{12} &\dots & a_{1n}\\
a_{21} & a_{22} &\dots & a_{2n}\\
\vdots & \vdots && \vdots\\
a_{m1} & a_{m2} &\dots & a_{mn}\\
\end{pmatrix}
$$
for an $n\times 1$ vector $\va$ and an $m\times n$ matrix $\mA$.
If one has a choice, we recommend that $m\geq n$.

We write $\mA=(a_{ij})$ or $\mA=(\mA)_{ij}$ to denote a typical
element of the matrix $\mA$.
The $n$ columns of $\mA$ are denoted by
$\va_{\bcdot 1}$, $\va_{\bcdot 2}$, \dots, $\va_{\bcdot n}$,
and the $m$ rows by
$\va'_{1\bcdot}$, $\va'_{2\bcdot}$, \dots, $\va'_{m\bcdot}$,
where transpose is denoted by a prime.
The symbol $\bcdot$ is produced by \type{bcdot} since
\type{cdot} ($\cdot$) is too small
and \type{bullet} ($\bullet$) is too large.
Hence,
$$
\mA=\left(\va_{\bcdot 1}, \va_{\bcdot 2}, \dots, \va_{\bcdot n}\right),
\qquad
\mA'=\left(\va_{1\bcdot}, \va_{2\bcdot}, \dots, \va_{m\bcdot}\right).
$$
However, we may write
$\mA=\left(\va_{1}, \va_{2}, \dots, \va_{n}\right)$,
and occasionally
$\mA'=\left(\va_{1}, \va_{2}, \dots, \va_{m}\right)$,
when there is no possibility of confusion,
but not both of these in one paper.
A vector $\va$ denotes a column and $\va'$ denotes a row.
Special vectors are:
\begin{tabbing}
\hspacesymbols \= \hspacesymbols\quad \=  \kill
$\vzeros$, $\vzeros_n$ \> null vector $(0,0,\dots,0)'$ (\type{vzeros})\\
$\vones$, $\vones_n$ \> sum vector $(1,1,\dots,1)'$  (\type{vones})\\
$\ve_i$ \> $i$-th column of $\mI_n$ (\type{ve\_i}).
\end{tabbing}
Special matrices are:
\begin{tabbing}
\hspacesymbols \= \hspacesymbols\qquad\quad \= \kill
$\mZeros$, $\mZeros_{mn}$ \> null matrix of order $m\times n$ (\type{mZeros}) \\
$\mI$, $\mI_n$ \> identity matrix of order $n\times n$ (\type{mI}).
\end{tabbing}
Note that the null vector $\vzeros$ is smaller
than the null matrix $\mZeros$.
We say that two or more matrices (vectors) are {\it conformable\/}
if their sum or product is defined. For example, the equation
$\mA\vx=\vb$ only makes sense if the dimension of $\vx$ equals the number
of columns of $\mA$ and the dimension of $\vb$ equals the number
of its rows.
If this is the case then $\mA$, $\vx$ and $\vb$ are conformable.

Two vectors $\va$ and $\vb$ for which $\va'\vb=0$ are {\it orthogonal\/}.
We also write $\va\bot\vb$ (\type{bot}).
The column space of $\mA$ is denoted $\col(\mA)$ (\type{col}) and denotes
the set $\{\vx: \vx=\mA\vc \text{ for some } \vc\ne \vzeros\}$.
The null space of $\mA$ is the set $\{\vx:\mA\vx=\vzeros\}$.
The null space of $\mA'$ is denoted $\col^{\bot}(\mA)$
and is called the {\it orthogonal complement\/} of $\col(\mA)$.
It defines the set $\{\vx:\mA'\vx=\vzeros\}$,
which can also be written as $\{\vx:\vx\bot\mA\}$.
%
\section{Operations on matrix $A$ and vector $a$}
The following standard operations are proposed.
\begin{tabbing}
\hspacesymbols \= \hspacesymbols\qquad\quad \=  \kill
$\mA'$ \> transpose   \\
$\mA^{-1}$ \> inverse   \\
$\mA^{+}$ \> Moore-Penrose inverse   \\
$\mA^{-}$ \> generalized inverse (satisfying $\mA\mA^{-}\mA=\mA$)   \\
$\dg \mA$, $\dg(\mA)$ \> diagonal matrix containing the diagonal elements \\
\> of $\mA$ (\type{dg})   \\
$\diag(a_1,\dots,a_n)$ \> diagonal matrix containing $a_1,\dots,a_n$ \\
\> on the diagonal (\type{diag})  \\
$\diag(\mA_1,\dots,\mA_n)$ \> block-diagonal matrix with $\mA_1,\dots,\mA_n$ on
the diagonal \\
$\mA^2$ \> $\mA\mA$  \\
$\mA^{1/2}$ \> (unique) square root of positive semidefinite matrix  \\
$\mA^p$ \> $p$-th power  \\
$\mA^{\#}$ \> adjoint (matrix)\\
$\mA^*$ \> conjugate transpose\\
\> (If $\mA:=\mU + \iu\mV$ then $\mA^*=\mU' - \iu\mV'$)\\
$\mA_{k}$ \> principal submatrix of order $k\times k$\\
$(\mA, \mB)$, $(\mA:\mB)$ \> partitioned matrix   \\
$\vec \mA$, $\vec(\mA)$ \> vec operator (\type{vec})  \\
$\vech \mA$, $\vech(\mA)$ \> vector containing $a_{ij}\,(i\geq j)$ (\type{vech})\\
$\rk(\mA)$ \> rank (\type{rk})   \\
$\lambda_{i}$, $\lambda_{i}(\mA)$ \> $i$-th eigenvalue (of $\mA$)\\
$\tr \mA$, $\tr(\mA)$ \> trace (\type{tr})   \\
$\etr \mA$, $\etr(\mA)$ \> $\exp(\tr \mA)$ (\type{etr}) \\
$|\mA|$, $\det \mA$, $\det(\mA)$ \> determinant (\type{det}) \\
$\|\mA\|$ \> norm of matrix ($\sqrt{(\tr \mA^*\mA)}$) (\type{|}) \\
$\|\va\|$ \> norm of vector ($\sqrt{(\va^*\va)}$)\\
$\mA\geq \mB$, $\mB\leq \mA$ \> $\mA-\mB$ positive semidefinite
(\type{geq}, \type{leq}\type{le})\\
$\mA>\mB$, $\mB<\mA$ \> $\mA-\mB$ positive definite ($>$, $<$)\\
$\mA\otimes\mB$ \> Kronecker product (\type{otimes})  \\
$\mA\odot\mB$ \> Hadamard product (\type{odot})\\
$\mK_{mn}$ \> commutation matrix    \\
$\mK_n$ \> $\mK_{nn}$ \\
$\mN_n$ \> $\frac{1}{2}(\mI_{n^2}+\mK_n)$   \\
$\mD_n$ \> duplication matrix    \\
$\mJ_{k}(\lambda)$ \> Jordan block of order $k\times k$
\end{tabbing}
Ambiguity can arise between the symbol $|\cdot|$ for determinant and
the same symbol for absolute value, for example in the
multivariate transformation theorem.
This ambiguity can be avoided by writing $|\det A|$
for the absolute value of a determinant.

If we have a symmetric matrix $\mA$ of order $n\times n$, then
the eigenvalues are real and can be ordered.
We recommend the ordering
$$
\lambda_1\geq\lambda_2\geq\dots\geq\lambda_n,
$$
since there are many cases where it is desirable that
$\lambda_1$ denotes the largest eigenvalue.
%
\section{The linear regression model}
We write the linear regression model
$\vy=\mX\vbeta +\vepsi$
as
$$
\vy=\sum_{h=1}^k \beta_h \vx_{\bcdot h} + \vepsi
$$
or as
$$
y_i = \vx'_{i\bcdot}\vbeta + \epsi_i\quad (i=1,2,\dots,n)
$$
or as
$$
y_i = \beta_1 x_{i1} + \beta_2 x_{i2} + \dots + \beta_k x_{ik} + \epsi_i\quad (i=1,2,\dots,n).
$$
If there is a constant term this specializes to
$$
y_i = \beta_1 + \beta_2 x_{i2} + \dots + \beta_k x_{ik} + \epsi_i\quad (i=1,2,\dots,n).
$$
In the two-variable case one can write
$$
y_i = \beta_1 + \beta_2 x_{i} + \epsi_i
\quad
\text{or}
\quad
y_i = \alpha + \beta x_{i} + \epsi_i,
$$
but {\it not\/}
$y_i = \beta_0 + \beta_1 x_{i} + \epsi_i$,
since $\beta_0$ is often used for other purposes,
in particular as the value of the parameter
$\beta$ under the null hypothesis.

The observations are typically indexed $i=1,\dots,n$ (in cross sections) or
$t=1,\dots,T$ (in time series). If there are two cross sections one can
use $i$ and $j$; if there are two time series one uses $t$ and $s$.
There are $k$ regressors (not $K$) indexed by $h=1,\dots,k$.
Acronyms and special symbols take precedence over index labels.
For example, in defining the $t$-statistic one should not
use $t$ as a summation index, and in formulae involving the imaginary unit
$\iu$ confusion can be avoided by not using $i$ as an index.

This formulation is not without controversy.
Some authors write
$X_{ht}$ instead of $x_{ih}$, which is unsatisfactory, since $\mX$ is an
$n\times k$ matrix and hence in their formulation $X_{ht}$ is the $th$-th
element of $\mX$.
Some write $\beta_0$ for the first element of $\vbeta$, if the
regression contains a constant term, and then let $k$ denote the number
of `real' regressors (so that $\mX$ has $k+1$ columns).
We prefer to avoid this formulation for many reasons. It is
convenient to always have $k$ regressors independent of whether there
is a constant term or not. Also, the inclusion of a constant
does make an important difference, for example in potentially
non-stationary time series, and it can translate into a `real' variable
such as a drift, which alters distributions and time paths.

Another issue is the disturbance term. We denote this by $\vepsi$
(\type{epsi} for a scalar, \type{vepsi} for a vector)
if the disturbances (or errors) are spherically distributed.\footnote{The
vector $\vepsi$ is spherically distributed if $\vepsi$
and $\mH\vepsi$ have identical distributions for every orthogonal
matrix $\mH$.}
If the errors are not spherical, we denote them by $\vu$.

Estimators are random variables which say something about a fixed but unknown
quantity, called a parameter.
They are denoted by `hats', such as $\hat{\vbeta}$.
(\type{hat\{$\backslash$vbeta\}}).
If we have a second estimator of $\vbeta$ this is denoted by a `tilde':
$\tilde{\vbeta}$. The realization of an estimator is an estimate.

Predictors are like estimators, except that they say something about a random variable.
They are also denoted by `hats' ($\hat{\vy}$, $\hat{\vepsi}$)
or tildes ($\tilde{\vy}$, $\tilde{\vepsi}$).
The realization of a predictor is the `prediction'.

The symbols
$R^2$ and $\bar{R}^2$ denote the coefficient of determination
and the adjusted coefficient of determination, respectively.

In the case of OLS (ordinary least squares), it is tradition to write
$\vb$ instead of $\hat{\vbeta}$ for the OLS estimator $(\mX'\mX)^{-1}\mX'\vy$,
$\ve$ instead of $\hat{\vepsi}$ for the residuals,
and $s^2$ instead of $\hat{\sigma}^2$ for the OLS estimator of
$\sigma^2$.\footnote{In line with current practice, we write the estimator for
$\sigma^2$ as $\hat{\sigma}^2$
(\type{hat\{$\backslash$sigma\}})
and not as $\hat{\sigma^2}$, although strictly speaking
the latter is the correct notation.}
We prefer not to do so, in order to stress the randomness of the estimators
(one often thinks of $\vb$ as a vector of constants).

If $\va$ is a vector, say of order $n$,
then $\bar{a}$ (\type{bar})
denotes the average of its components:
$\bar{a}=\vones'\va/\vones'\vones$.

It is customary to write
$$
\mP_{\mX}=\mX(\mX'\mX)^+\mX',\qquad
\mM_{\mX}=\mI_n - \mP_{\mX}
$$
where $\mX$ has $n$ rows. If there is no possibility of confusion,
we can write $\mM$ and $\mP$ instead of $\mM_{\mX}$ and $\mP_{\mX}$.
The matrix which puts a vector in deviation form is thus
$$
\mM_{\vones}=\mI_n-(1/n)\vones\vones',
$$
and the vector $\mM_{\vones}\va$
denotes the vector $\va$ in deviation from its mean.

We denote a null hypothesis as $\rH_0$ (\type{rH}) and an alternative
as $\rH_A$ (not $\rH_a$ since $a$ may be a scalar
or may refer to `asymptotic').
The statement of $\rH_0$:
$\mR'\vbeta=\vc$ is preferred over
$\mR\vbeta=\vr$. In the latter formulation,
the single-hypothesis case is usually written as
$\vw'\vbeta=r$ or $\vr'\vbeta=r$, neither of which is ideal.
However, if one writes
$\mR'\vbeta=\vc$, this specializes to $\vr'\vbeta=c$
in the one-dimensional case.
This has the additional advantage that we can use $r$ to denote the
number of restrictions (dimension of $\vc$).
In the special case where $\mR=\mI_r$ (or where $\mR$ is square and
invertible), we usually write $\vbeta=\vbeta_0$ rather than
$\vbeta=\vc$.

The GLS model is written
$$
\vy=\mX\vbeta + \vu,\qquad \vu\distr \text{N}(\vzeros,\mOmega),
$$
where $\mOmega=\sigma^2\mV$.
We prefer the use of $\mOmega$ over $\mSigma$,
which can be confused with the summation symbol.
If the variance matrix $\mOmega$ contains parameters, we write
$\mOmega=\mOmega(\valpha)$. The dimension of $\valpha$ is $l$.
Thus, the whole model contains $m=k+l$ unknown parameters.
The complete parameter vector is denoted $\vtheta=(\vbeta',\valpha')'$
and is of dimension $m$.

For the {\it simultaneous equations model\/} our starting point is the
(univariate) linear regression model
$$
y_i = \vx'_{i\bcdot}\vbeta + u_i\quad (i=1,2,\dots,n).
$$
This can be generalized to the {\it multivariate\/} linear regression
model:
$$
\vy'_{i\bcdot} = \vx'_{i\bcdot}\mB + \vu'_{i\bcdot}\quad (i=1,2,\dots,n),
$$
where $\vy_{i\bcdot}$ and $\vu_{i\bcdot}$ are random $m\times 1$ vectors
and $\mB$ is a $k\times m$ matrix.
The univariate case is obtained as a special case when $m=1$.
The simultaneous equations model provides a further generalization:
$$
\vy'_{i\bcdot}\mGamma = \vx'_{i\bcdot}\mB + \vu'_{i\bcdot}\quad (i=1,2,\dots,n),
$$
where $\mGamma$ is an $m\times m$ matrix.
This is the {\it structural form\/} of the simultaneous equations model.
In matrix notation this becomes
$\mY\mGamma = \mX\mB + \mU$. If $\mGamma$ is invertible, we obtain
the {\it reduced form\/}
$\mY=\mX\mPi + \mV$, where
$\mPi=\mB\mGamma^{-1}$ and $\mV=\mU\mGamma^{-1}$.
%
\section{Greek symbols}
Some Greek lowercase letters have variant forms and these can
be used to mean different things than the usual letter.
We have:
\begin{tabbing}
\qquad\qquad \= \qquad\qquad\qquad\qquad\qquad\qquad \= \qquad\qquad \= \qquad\qquad \=  \kill
$\eps$ \> \type{epsilon}, \type{eps} \> $\varepsilon$ \> \type{varepsilon}, \type{epsi}  \\
$\theta$ \> \type{theta} \> $\vartheta$ \> \type{vartheta} \\
%$\kappa$ \> \type{kappa} \> $\varkappa$ \> \type{varkappa} \\
$\pi$ \> \type{pi} \> $\varpi$ \> \type{varpi} \\
$\rho$ \> \type{rho} \> $\varrho$ \> \type{varrho} \\
$\sigma$ \> \type{sigma} \> $\varsigma$ \> \type{varsigma} \\
$\phi$ \> \type{phi} \> $\varphi$ \> \type{varphi}
\end{tabbing}
We shall use $\epsi$ (\type{epsi} for a scalar, \type{vepsi} for a vector)
for a disturbance term and $\eps$ (\type{eps}) for
an arbitrarily small positive number.
Also, we use $\theta$ (\type{theta}) to denote a variable
and $\vartheta$ (\type{vartheta}) for a function.
%
\section{Mathematical symbols, functions and operators}
Definitions, implications, convergence, and transformations are denoted by
\begin{tabbing}
\hspacesymbols \= \hspacesymbols\qquad\quad \=  \kill
$\equiv$ \> identity, equivalence (\type{equiv}) \\
$a:=b$ \> defines $a$ in terms of $b$ \\
$\Longrightarrow$ \> implies (\type{implies})\\
$\Longleftrightarrow$ \> if and only if (\type{iff})\\
$\to$, $\longto$ \> converges to (\type{to}, \type{longto}) \\
$x\mapsto y$ \> transformation from $x$ to $y$ (\type{mapsto})
\end{tabbing}
We write $f(x)\approx g(x)$ (\type{approx}) if the two functions
are approximately equal in some sense depending on the context.
If $f(x)$ is proportional to $g(x)$ we write
$f(x)\propto g(x)$ (\type{propto}).
We say that `$f(x)$ is at most of order $g(x)$' and write
$f(x)=O(g(x))$, if $|f(x)/g(x)|$ is bounded above in some
neighborhood of $c$ (possibly $\pm\infty$), and we say that
`$f(x)$ is of order less than $g(x)$' and write
$f(x)=o(g(x))$, if $f(x)/g(x)\to 0$ when $x\to c$.
Finally, we write $f(x)\sim g(x)$ (\type{sim}) if
$f(x)/g(x)\to 1$ when $x\to c$. The two functions are then said
to be `asymptotically equal'.\footnote{The ISO
prescribes the symbol $\simeq$ (\type{simeq}) for asymptotic equality,
but $\sim$ is common practice in econometrics and statistics,
even though the same symbol is also
used for `is distributed as'.}
Notice that when $f(x)$ and $g(x)$ are asymptotically equal, then
$f(x)\approx g(x)$ and also $f(x)=O(g(x))$, but not vice versa.

For example, when $\phi$ and $\Phi$ denote the p.d.f. and c.d.f. of the
standard-normal distribution, respectively, we write
the leading term (first term) of the asymptotic expansion
$$
\frac{\Phi(x)}{\phi(x)}\sim \frac{1}{|x|}\qquad \text{as } x\to -\infty.
$$
However, there are many good local approximations of this ratio which
are not necessarily asymptotically equal to it.

The usual sets are denoted as follows:
\begin{tabbing}
\hspacesymbols \= \hspacesymbols\qquad\quad \=  \kill
$\SN$ \> natural numbers $1,2,\dots$ \> \type{SN} \\
$\SZ$ \> integers $\dots,-2,-1,0,1,2,\dots$ \> \type{SZ} \\
$\SQ$ \> rational numbers \> \type{SQ} \\
$\SR$ \> real numbers \> \type{SR} \\
$\SC$ \> complex numbers \> \type{SC}
\end{tabbing}
Superscripts denote the dimension and subscripts the relevant subset.
For example,
$\SR^2 = \SR\times\SR$ denotes the real plane,
$\SR^n$ the set of real $n\times 1$ vectors, and
$\SR^{m\times n}$ the set of real $m \times n$ matrices.
The set $\SR^n_{+}$ denotes the positive orthant of $\SR^n$, while
$\SZ_+$ denotes the set of positive integers (hence, $\SZ_+=\SN$) and
$\SZ_{0,+}$ denotes the non-negative integers.
Finally, $\SC^{n\times n}$ denotes the set of complex $n\times n$ matrices.

Set differences are denoted by a backslash (\type{backslash}).
For example, $\SN=\SZ_{0,+}\backslash\{0\}$.
Real-line intervals defined by $x$ in $a\leq x<b$ are denoted by
$[a,b)$. Occasionally it might be unclear whether $(a,b)$ indicates a real-line
interval or a point in $\SR^2$. In that case the interval
$a<x<b$ can alternatively be written as $]a,b[$.

Sequences are special ordered sets. They are delimited, as usual,
by braces (curly brackets). It is often convenient to write
$\{\mZ_j\}_m^n$ (or simply $\{\mZ_j\}$) for the sequence of matrices
$\mZ_m$, $\mZ_{m+1}$,\dots,$\mZ_n$.

Other symbols used are:
\begin{tabbing}
\hspacesymbols \= \hspacesymbols\qquad\quad \=  \kill
$\in$ \> belongs to (\type{in}) \\
$\notin$ \> does not belong to (\type{notin}) \\
$\{x: x \in S, x$ satisfies $P\}$ \> set of all elements of $S$ with
property $P$ \\
$\subseteq$ \> is a subset of (\type{subseteq}) \\
$\subset$ \> is a proper subset of (\type{subset}) \\
$\cup$ \> union  (\type{cup}) \\
$\cap$ \> intersection  (\type{cap}) \\
$\emptyset$ \> empty set (\type{emptyset})  \\
$A^{c}$ \> complement of $A$   \\
$B\backslash A$ \> $B\cap A^c$  \\
%$\overset{\circ}{S}$ \> interior of $S$ (\type{overset})\\
$\interior{S}$ \> interior of $S$ (\type{interior\{S\}})   \\
$S'$ \> derived set of $S$   \\
$\bar{S}$ \> closure of $S$ (\type{bar\{S\}})  \\
$\partial S$ \> boundary of $S$ (\type{partial S})
\end{tabbing}
We denote functions by
\begin{tabbing}
\hspacesymbols \=   \kill
$f: S \to T$ \> function defined on $S$ with values in $T$   \\
$f$, $g$, $\varphi$, $\psi$, $\vartheta$ \> scalar-valued function   \\
$\vf$, $\vg$ \> vector-valued function   \\
$\mF$, $\mG$ \> matrix-valued function   \\
$\vg \circ \vf$, $\mG \circ \mF$ \> composite function (\type{circ}) \\
$g*f$ \> convolution $(g*f)(x)=\int_{-\infty}^{\infty}g(y)f(x-y)\rd y$
\end{tabbing}
For their differentials, derivatives and differences, we write
\begin{tabbing}
\hspacesymbols \=   \kill
$\rd$ \> differential (\type{rd})  \\
$\rd^n$ \> $n$-th order differential   \\
$\rD_{j}\varphi(\vx)$ \> partial derivative (\type{rD}),
$\partial\varphi(\vx)/\partial x_j$ \\
$\rD_{j}f_i(\vx)$ \> partial derivative,
$\partial f_i(\vx)/\partial x_j$ \\
$\rD_{kj}^2\varphi(\vx)$ \> second-order partial derivative,
$\partial\rD_{j}\varphi(\vx)/\partial x_k$ \\
$\rD_{kj}^2 f_i(\vx)$ \> second-order partial derivative,
$\partial\rD_j f_i(\vx)/\partial x_k$ \\
$\varphi^{(n)}(x)$ \> $n$-th order derivative of $\varphi(x)$ \\
$\rD\varphi(\vx)$, $\partial\varphi(\vx)/\partial \vx'$ \> derivative of $\varphi(\vx)$ \\
$\rD \vf(\vx)$, $\partial \vf(\vx)/\partial \vx'$ \> derivative (Jacobian matrix) of $\vf(\vx)$ \\
$\rD \mF(\mX)$ \> derivative (Jacobian matrix) of $\mF(\mX)$  \\
$\partial\vec \mF(\mX)/\partial(\vec \mX)'$ \> derivative of $\mF(\mX)$, alternative notation\\
$\nabla\varphi$, $\nabla \vf$, $\nabla \mF$ \> gradient (transpose of derivative) (\type{nabla})\\
$\rH\varphi(\vx)$, $\partial^2\varphi(\vx)/\partial \vx\partial \vx'$ \>
second derivative (Hessian matrix) of $\varphi(\vx)$ (\type{rH})\\
$\rL$, $\rB$ \> backward shift operator: $\rL x_t=x_{t-1}$  (\type{rL},\type{rB})\\
$\diff$ \> (backward) difference operator: \\
\>$\diff x_t=x_t-x_{t-1}$  (\type{diff})\\
$\fordiff$ \> forward difference operator: \\
\> $\fordiff x_t=x_{t+1}-x_t$ (\type{fordiff})\\
$[f(x)]_a^b$,\; $f(x)|_a^b$ \> $f(b)-f(a)$
\end{tabbing}
Instead of $\varphi^{(1)}(x)$ and $\varphi^{(2)}(x)$, one can
write the more common
$\varphi'(x)$ and $\varphi''(x)$, but otherwise we prefer to reserve
the prime for matrix transposes rather than derivatives. Notice the
difference between the differencing operator \type{diff} ($\diff$)
and the gradient \type{nabla} ($\nabla$).

We use $\rL$ (or $\rB$) rather than $\calL$ for the lag operator in order to avoid
confusion with the Laplace transform. This and
other useful transforms are defined by
\begin{tabbing}
\hspacesymbols \= \hspacesymbols\qquad\quad \=  \kill
$\calF\{\cdot\}$ \> Fourier transform (\type{calF}) \\
$\calF^{-1}\{\cdot\}$ \> inverse Fourier transform \\
$\calL\{\cdot\}$ \> Laplace transform (\type{calL}) \\
$\calL^{-1}\{\cdot\}$ \> inverse Laplace transform \\
$\calM\{\cdot\}$ \> Mellin transform (\type{calM}) \\
$\calM^{-1}\{\cdot\}$ \> inverse Mellin transform
\end{tabbing}
Finally, various other symbols in common use are
\begin{tabbing}
\hspacesymbols \=   \kill
$\iu$ \> imaginary unit (\type{iu}) \\
${\eu}$, $\exp$ \> exponential (\type{eu}, \type{exp}) \\
$\log$ \> natural logarithm (\type{log})\\
$\log_a$ \> logarithm to the base $a$ \\
$!$ \> factorial \\
$\delta_{ij}$ \> Kronecker delta \\
$\sgn(x)$ \> sign of $x$ (\type{sgn}) \\
$\lfloor x\rfloor$, $\ip(x)$ \> integer part of $x$, that is,
largest integer $\leq x$ \\
\> (\type{lfloor}, \type{rfloor}, \type{ip}) \\
$|x|$ \> absolute value (modulus) of scalar $x\in\SC$ \\
$x^*$ \> complex conjugate of scalar $x\in\SC$ \\
$\Re(x)$ \> real part of $x$  (\type{Re}) \\
$\Im(x)$ \> imaginary part of $x$  (\type{Im}) \\
$\varGamma(x)$ \> gamma (generalized factorial) function,\\
\> satisfying $\varGamma(x+1)=x\varGamma(x)$ \\
$B(x,y)$ \> beta function, $\varGamma(x)\varGamma(y)/\varGamma(x+y)$ \\
$1_{\calK}$ \> indicator function (use $1$, not $I$): \\
\> equals 1 if condition $\calK$ is satisfied, 0 otherwise\\
$B(\vc)$, $B(\vc;r)$, $B(\mC;r)$ \> neighborhood (ball) with center $\vc$ ($\mC$)
and radius $r$\\
$\calV^{n\times k}$ \> Stiefel manifold: set of real $n\times k$
matrices $\mX$ \\
\>such that $\mX'\mX=\mI_k$\, ($k\le n$) (\type{calV}) \\
$\calO^n$ \> $\calV^{n\times n}$, orthogonal group of dimension $n$ (\type{calO}) \\
$\calO_+^n$ \> proper orthogonal group of dimension $n$ \\
\>(orthogonal $n\times n$ matrices with determinant $+1$) \\
$\calS^n$ \> $\calV^{n\times 1}$, unit sphere in $\SR^n$ (\type{calS})
\end{tabbing}
The Stiefel manifold $\calV^{n\times k}$ is also denoted
as $\calV^{k\times n}$ in the literature. We recommend the former
notation which is in line with $\SR^{n\times k}$.
%
\section{Statistical symbols, functions and operators}
The following symbols are commonly used.
\begin{tabbing}
\hspacesymbols \= \hspacesymbols\qquad\quad \=  \kill
$\distr$ \> is distributed as (\type{distr})  \\
$\adistr$ \> is asymptotically distributed as (\type{adistr}) \\
$\Pr(\cdot)$ \> probability (\type{Pr})  \\
$\E(\cdot)$ \> expectation (\type{E})  \\
$\E(\cdot|\cdot)$ \> conditional expectation \\
$\var(\cdot)$ \> variance (matrix) (\type{var})   \\
$\cov(\cdot,\cdot)$ \> covariance (matrix) (\type{cov})   \\
$\corr(\cdot,\cdot)$ \> correlation (matrix) (\type{corr})  \\
$L(\vtheta)$ \> likelihood function \\
$\ell(\vtheta)$ \> log-likelihood function (\type{ell}) \\
$\vq(\vtheta)$ \> score vector \\
$\Hesmat(\vtheta)$ \> Hessian matrix (\type{Hesmat}) \\
$\Infmat$ \> (Fisher) information matrix (\type{Infmat})   \\
$\calF_t$ \> filtration at time $t$ (\type{calF})\\
$t$ \> $t$-statistic, $t$-value \\
$\to$, $\longto$ \> converges a.s. (\type{to}, \type{longto}) \\
$\pto$ \> converges in probability (\type{pto}) \\
$\dto$ \> converges in distribution (\type{dto}) \\
$\wto$ \> converges weakly (\type{wto}) \\
$\plim$ \> probability limit (\type{plim}) \\
$O_p(g(x))$ \> at most of probabilistic order $g(x)$ \\
$o_p(g(x))$ \> of probabilistic order less than $g(x)$
\end{tabbing}
Notice that the symbol $\to$ ($\longto$) indicates both convergence
and a.s. convergence.
The symbol $\wto$ for weak convergence is preferred to
$\Longrightarrow$, which denotes logical implication.
The matrix $-\Hesmat$ is also called the observed information matrix,
while its expectation
$\Infmat:=-\E(\Hesmat)$ is the expected information matrix.

The main distributions in statistics are denoted as follows.
\begin{tabbing}
\hspacesymbols \= \hspacesymbols\qquad\quad \=  \kill
$\Bin(n,p)$ \> binomial distribution (\type{Bin}) \\
$\Poi(\mu)$ \> Poisson distribution (\type{Poi}) \\
$\rU(a,b)$ \> uniform distribution (\type{rU}) \\
$\rN_{m}(\vmu,\mOmega)$ \> $m$-dimensional normal distribution (\type{rN})  \\
$\LN(\mu,\sigma^2)$ \> lognormal distribution (\type{LN}) \\
$\phi(\cdot)$ \> standard-normal p.d.f. (\type{phi}) \\
$\Phi(\cdot)$ \> standard-normal c.d.f. \\
$\text{IN}_m(\vmu_i,\mOmega_i)$ \> sequence $i=1,2,\dots$ of independent \\
\>$m$-dimensional normal distributions \\
$\chi^2_{n}(\delta)$ \> chi-squared distribution with $n$ d.f.\\
\>and non-centrality parameter $\delta$. \\
$\chi^2_{n}$ \> central chi-squared ($\delta=0$) \\
$\rt_n(\delta)$ \> Student distribution with $n$ d.f. and \\
\>noncentrality $\delta$ (\type{rt}) \\
$\rt_n$ \> central t ($\delta=0$) \\
$\rC(a,b)$ \> Cauchy distribution (\type{rC}) \\
$\rF_{m,n}(\delta)$ \> Fisher distribution with $m$ (numerator) and \\
\>$n$ (denominator) d.f. and non-centrality $\delta$ (\type{rF})\\
$\rF_{m,n}$ \> central F ($\delta=0$) \\
$\Gamma(\alpha,\lambda)$ \>  gamma distribution \\
$\rB(a,b)$ \> beta distribution (\type{rB}) \\
$W(\tau)$, $B(\tau)$ \> standard Wiener process (Brownian motion)\\
\>on $\tau\in [0,1]$
\end{tabbing}
We use the word `expectation' to denote mathematical expectation of a
random vector $\vx$, written $\E(\vx)$. The word `average' refers to
taking the average of some numbers:
$\bar{x}=(1/n)\sum_{i=1}^n x_i$. The word `mean' which could indicate
either is best avoided. Like `expectation', the words `variance' ($\var$),
`covariance' ($\cov$), and
`correlation' ($\corr$)
indicate population parameters. The corresponding
sample parameters are called `sample variance', `sample covariance' and
`sample correlation'.

The `standard deviation' is the positive square root of the variance.
If $\theta$ is a parameter which we estimate by $\hat{\theta}$,
then this estimator is a random variable with a variance
$\var(\hat{\theta})$
and a standard deviation
$\sqrt{\var(\hat{\theta})}$.
In general, this standard deviation depends on unknown parameters.
Both the estimator of the standard deviation
and its realization are called the `standard error'.
The $t$-statistic is a random variable (not necessarily Student
distributed); its realization is the $t$-value.
%
\section{Abbreviations and acronyms}
\begin{tabbing}
\hspacesymbols \= \hspacesymbols\qquad\quad \=  \kill
2SLS \> two-stage least squares \\
3SLS \> three-stage least squares \\
AR($p$) \> autoregressive process of order $p$ \\
ARCH \> autoregressive conditional heteroskedasticity \\
ARIMA($p,d,q$)\> autoregressive integrated moving-average process \\
ARMA($p,q$) \> autoregressive moving-average process \\
a.s. \> almost surely   \\
BAN \> best asymptotically normal \\
c.d.f. \> cumulative distribution function \\
c.f. \> characteristic function \\
c.g.f. \> cumulant-generating function \\
CLT \> central limit theorem \\
CUAN \> consistent uniformly asymptotically normal \\
d.f. \> degrees of freedom \\
DW \> Durbin-Watson \\
FCLT \> functional CLT (invariance principle) \\
FGLS \> feasible generalized least squares \\
FIML \> full-information maximum likelihood \\
f.m.g.f. \> factorial moment-generating function \\
GLS \> generalized least squares \\
GMM \> generalized method of moments \\
i.i.d. \> independent and identically distributed \\
ILS \> indirect least squares \\
I($d$) \> (fractionally) integrated process of order $d$ \\
IV \> instrumental variable \\
LAD \> least absolute deviations \\
LIL \> law of iterated logarithm \\
LIML \> limited-information maximum likelihood \\
LLN \> law of large numbers \\
LM \> Lagrange multiplier \\
LR \> likelihood ratio \\
LS[E] \> least squares [estimator]; see also 2SLS, 3SLS,\\
\> FGLS, GLS, ILS, NLS, OLS, RLS  \\
MA($q$) \> moving-average process of order $q$ \\
m.g.f. \> moment-generating function \\
ML[E] \> maximum likelihood [estimator]; \\
\> see also FIML, LIML, QML  \\
MSE \> mean squared error   \\
NLS \> nonlinear least squares \\
OLS \> ordinary least squares \\
p.d.f. \> probability density function \\
QML[E] \> quasi-maximum likelihood [estimator] \\
RLS \> restricted least squares \\
r.v. \> random variable \\
s.e. \> standard error \\
SUR \> seemingly unrelated regression\\
UMP \> uniformly most powerful \\
W \> Wald
\end{tabbing}
%
\section{Hopes, fears and expectations}
Our hope is that this paper may contribute towards the establishment
of a common notation in econometrics.
Our fear is that it will not.
We realize that it will be difficult to get consensus.
The $=$ sign for equality was first proposed
in the middle of the 16th century, but 150 years later Bernoulli
still used $\propto$ (stylized \ae, short for {\it aequalis\/})
in his {\it Ars Conjectandi\/}.
Thus, our expectation is that it could take 150 years before a
common notation is adopted.
%
\section*{Acknowledgements}
We are grateful to Paolo Paruolo for helpful comments.
%
\section*{References}
Beccari, C. (1997). Typesetting mathematics for science and technology
according to ISO 31/XI, {\it TUGboat 18}, No. 1, 39--48.\\
\ \\
Bollob\'{a}s, B. (ed.) (1986). {\it Littlewood's Miscellany\/}, Cambridge:
Cambridge University Press.\\
\ \\
Kline, M. (1972). {\it Mathematical Thought From Ancient to Modern Times\/},
New York: Oxford University Press.\\
\ \\
Magnus,~J. R. and M. S.~Morgan (1987). The ET interview: Professor
J.~Tinbergen, {\it Econometric Theory 3}, 117--142.\\
\ \\
Mathematical Signs and Symbols for Use in the Physical Sciences and
Technology, ISO 31/XI (1982). In {\it ISO Standards Handbook 2\/},
186--214, Geneva: International Organization for Standardization.\\
\ \\
Symbols, Units and Nomenclature in Physics (1988). In R. C.~West (ed.),
{\it CRC Handbook of Chemistry and Physics\/}, F156--F193, 1st Student Edition,
Boca Raton, Florida: CRC Press, Inc.
\end{document}
